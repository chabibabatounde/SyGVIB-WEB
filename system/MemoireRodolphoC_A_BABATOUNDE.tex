\documentclass[12pt]{report}
\usepackage{style_ifri}
\usepackage{array}
\usepackage{colortbl}
\usepackage{tikz}
\usepackage{shorttoc}
\usepackage{abstract,lipsum}
\usepackage{indentfirst}
\usepackage{wasysym}

\titleformat{\section}[wrap]
{\normalfont\bfseries}
{\thesection.}{0.5em}{}

\typeMemoire{Master en Informatique}
%\optionFormation{Informatique Générale}
\etudiant{M. Rodolpho Chabi A. \textbf{BABATOUNDE}}{chabibabatounde@gmail.com}

\titreDuMemoire{Conception d'un système d'information de gestion du parc automobile béninois basé sur la lecture automatique des plaques
minéralogiques}

\anneeScolaire{2018-2019}


% Plusieurs encadrants décommenté la ligne suivante
\encadrants{\textit{Dr.} Ratheil V. \textbf{HOUDJI}}

% Un seul encadrant décommenté la ligne suivante
% \encadrants{Prof Eugène C. \textbf{EZIN}}

\renewenvironment{abstract}
{\begin{quote}
 \noindent \rule{\linewidth}{1pt}\par{\bfseries \abstractname.}}
 {\medskip \noindent \newline \rule{\linewidth}{1pt}
\end{quote}
}

\hypersetup{
	pdftitle={MEMOIRE MASTER INFORMATIQUE - RODOLPHO BABATOUNDE},
	pdfauthor={Rodolpho Chabi BABATOUNDE, chabibabatounde@gmail.com},
	pdfsubject={Conception d'un système d’information de gestion du parc automobile béninois basé sur la lecture automatique des plaques
minéralogiques},
	pdfkeywords={Systeme d'information, OCR, OpenCV, TensorFlow, Machine Learning, LAPI, Détection Immatriculation} 
}

\setcounter{tocdepth}{2}
% \setcounter{secnumdepth}{2}
%\loadglsentries[\newacronym]{glossaire/glossaire}

\begin{document}
%Décommenter ou commenter pour afficher ou non le logo en filigrane sur la 1ère page
% \AddToShipoutPicture*{\BackgroundPic}
\begin{onehalfspace}

\pageDeGarde

\selectlanguage{french}

% Sommaire
\pagenumbering{roman}
{
% Sommaire
% \dominitoc
\shorttableofcontents{Sommaire}{0}
\addcontentsline{toc}{chapter}{Sommaire}   %insertion ligne "sommaire" dans la table des matières
% \tableofcontents
% Dedicace
\input{glossaire/glossaire}
\include{dedicace/dedicace}
% Remerciements
\include{remerciements/remerciements}
\selectlanguage{french}
% Glossaire
\selectlanguage{french}
\glsaddall
\printglossaries\addcontentsline{toc}{chapter}{Liste des sigles et abréviations}
%\printglossary[type=\acronymtype , title={Liste des acronymes}, toctitle={Liste des acronymes}]
\clearpage
}
\clearpage
\pagenumbering{arabic}
\setcounter{page}{1}
% Résumé
\resume
\input{resume/resume}
\selectlanguage{french}
%introduction
\lhead[]{} \rhead[]{} \chead[]{}
%package auto-pst-pdf Error:"shell escape" (or "write18") is not enabled : auto-pst-pdf will not work
\fancyhead[L]{\tiny \leftmark}
\fancyhead[R]{\scriptsize \rightmark}
\fancyfoot[C]{\thepage}

% Introduction generale
\input{introduction/introduction}
\input{1-etat_art/etat_art}
\input{2-materielMethode/materielMethode}
\input{3-solution/solution}
\lhead[]{} \rhead[]{} \chead[]{}
%\conclusion
%\input{conclusion/conclusion}
% Bibliographie
%\include{bibliographie/bibliographie}
% Annexe
%\appendix
%\include{annexe/annexe}
% Table des matieres
\clearpage
\pagenumbering{gobble}
\setcounter{tocdepth}{1}
\tableofcontents
\addcontentsline{toc}{chapter}{Table des matières}
\end{onehalfspace}
\end{document}

